
\newpage
\section{Ejemplos}
\subsection{Códigos de Programación}
En esta sección se presenta la inserción del Código \ref{CodC}, El Código \ref{CodC}. El Código \ref{codP}. El Código \ref{codL}.



\lstset{language=C}
\begin{lstlisting}[caption = C\'odigo en C de una sumatoria, label = CodC]
#include <stdio.h>
#include <stdlib.h>
/* Algoritmo para realizar la sumatoria */
/* S=2+4+6+...+2n */

int main(void){
	int i,s,n;
	
	/* inicializar el valor de la sumatoria en 0 */
	s=0;
	printf("ingrese la cantidad de elementos de la sumatoria=");
	scanf("%d", &n);
	/* Realiza la iteracion n veces, y el indice "i" lo multiplica por */
	/* 2 y lo va sumando a s*/
	for(i=1;i<=n;i++){
		s = s+ 2*i;
	} 
	printf("el resultado de la sumatoria es=%d\n",s);

	return (0);
}
\end{lstlisting}

\lstset{language=SQL}
   \vspace{-0.8cm}
\begin{lstlisting}[label=codigo1, caption=Ejemplo de SQL]
-- Database: acuario

-- DROP DATABASE acuario;

CREATE DATABASE acuario
  WITH OWNER = postgres;


CREATE TABLE especies(
    sno integer PRIMARY KEY,
    nombre character varying(20),
    alimento character varying(20)
);

CREATE TABLE tanques(
    tno integer PRIMARY KEY,
    nombre_tanque character varying(20),
    color_tanque character varying(20),
    volumen  integer NOT NULL
);

CREATE TABLE peces(
    pno integer PRIMARY KEY,
    nombre_peces character varying(20),
    color_peces character varying(20),
    tno integer NOT NULL,
    sno integer NOT NULL,
    FOREIGN KEY (tno) REFERENCES tanques (tno) ON UPDATE CASCADE ON DELETE CASCADE,
    FOREIGN KEY (sno) REFERENCES especies (sno) ON UPDATE CASCADE ON DELETE CASCADE
);

CREATE TABLE eventos(
    eno integer PRIMARY KEY,
    pno integer NOT NULL,
    fecha date,
    FOREIGN KEY (pno) REFERENCES peces (pno) ON UPDATE CASCADE ON DELETE CASCADE
);



INSERT INTO especies VALUES(17,'delfin','arenque');
INSERT INTO especies VALUES(22,'tiburon','cualquier cosa');
INSERT INTO especies VALUES(74,'olomina','gusano');
INSERT INTO especies VALUES(93,'ballena','mantequilla de mani');
INSERT INTO especies VALUES(100,'pez espada','gusano');
INSERT INTO especies VALUES(120,'pez globo','gusano');

-- select * from especies

INSERT INTO tanques VALUES(55,'charco','verde',300);
INSERT INTO tanques VALUES(42,'letrina','azul',100);
INSERT INTO tanques VALUES(35,'laguna','rojo',400);
INSERT INTO tanques VALUES(85,'letrina','azul',100);
INSERT INTO tanques VALUES(38,'playa','azul',200);
INSERT INTO tanques VALUES(44,'laguna','verde',200);

-- select * from tanques


INSERT INTO peces VALUES (164, 'charlie', 'naranjo', 42, 74);
INSERT INTO peces VALUES (347, 'flipper', 'negro', 35, 17);
INSERT INTO peces VALUES (228, 'killer', 'blanco', 42, 22);
INSERT INTO peces VALUES (281, 'albert', 'rojo', 55, 17);
INSERT INTO peces VALUES (119, 'bonnie', 'azul', 42, 22);
INSERT INTO peces VALUES (388, 'cory', 'morado', 35, 93);
INSERT INTO peces VALUES (700, 'maureen', 'blanco', 44, 100);
INSERT INTO peces VALUES (800, 'beni', 'rojo', 55, 17);
INSERT INTO peces VALUES (900, 'nemo', 'rojo', 44, 74);
INSERT INTO peces VALUES (150, 'vicky', 'rojo', 55, 100);
INSERT INTO peces VALUES (160, 'mati', 'amarillo', 42, 100);
INSERT INTO peces VALUES (110, 'rafa', 'azul', 85, 100);
INSERT INTO peces VALUES (222, 'jimmy', 'amarillo', 38, 100);
INSERT INTO peces VALUES (144, 'bisho', 'rojo', 42, 93);
INSERT INTO peces VALUES (125, 'chris', 'azul', 38, 93);
INSERT INTO peces VALUES (183, 'sable', 'amarillo', 44, 93);
INSERT INTO peces VALUES (241, 'taz', 'rojo', 55, 93);
INSERT INTO peces VALUES (300, 'baltazar', 'azul', 85, 100);
INSERT INTO peces VALUES (200, 'cash', 'azul', 85, 100);
INSERT INTO peces VALUES (424, 'bandido', 'verde', 35, 100);
INSERT INTO peces VALUES (454, 'romo', 'blanco', 85, 93);


-- select * from peces

INSERT INTO eventos VALUES 
(3456 , 347 , '2010-01-26'),
(6653 , 164 , '2010-05-14'),
(5644 , 347 , '2010-05-15'),
(5645 , 347 , '2010-05-30'),
(6789 , 281 , '2010-04-30'),
(5211 , 228 , '2010-08-20'),
(6719 , 700 , '2010-10-22'),
(4555 , 164 , '2011-11-03'),
(9647 , 281 , '2011-12-06'),
(5347 , 281 , '2011-01-01');

--INSERT INTO eventos VALUES (3456, 164, '2010-01-26'); 
--INSERT INTO eventos VALUES (6653, 347, '2010-05-14'); 
--INSERT INTO eventos VALUES (5644, 347, '2010-05-15'); 
--INSERT INTO eventos VALUES (5645, 347, '2010-05-30'); 
--INSERT INTO eventos VALUES (6789, 228, '2010-04-30'); 
--INSERT INTO eventos VALUES (5211, 119, '2010-08-20'); 
--INSERT INTO eventos VALUES (6719, 388, '2010-10-22'); 
--INSERT INTO eventos VALUES (4555, 164, '2011-11-03'); 
--INSERT INTO eventos VALUES (9647, 281, '2011-12-21'); 
--INSERT INTO eventos VALUES (5369, 281, '2011-01-01'); 


-- ALTER TABLE tanques ADD medida character varying(2); 

-- UPDATE tanques SET medida = 'ml';

-- select * from tanques;

-- ALTER TABLE tanques DROP medida;

-- SELECT * FROM especies;
-- SELECT * FROM tanques;
\end{lstlisting}\vspace{-0.3cm}




\lstset{language=LISP}
\begin{lstlisting}[caption= C\'odigo LISP de una Lista, label = codL]
(define (length x)
    (if (list? x) (length-aux x)
        (error "x no es una lista")))
        
(define (length-aux x)
    (if (null? x) 0 (+1 (length-aux (cdr x)))))
\end{lstlisting}




\lstset{language=PROLOG}
\begin{lstlisting}[caption= C\'odigo PROLOG de un \'arbol geneal\'ogico, label=codP]
% Arbol genealogico version 1.
% padre(A,B) significa que B es el padre de A.

padre(juan,alberto).
padre(luis,alberto).
padre(alberto,leoncio). 
padre(geronimo,leoncio).
padre(luisa,geronimo). 

% Ahora se define las condiciones para que dos individuos sean hermanos hermano(A,B), significa que A es hermano de B.
hermano(A,B) :- 
    padre(A,P), 
    padre(B,P), 
    A \== B.
% Ahora se define el parentesco abuelo-nieto.  nieto(A,B) significa que A es nieto de B.
nieto(A,B) :- 
    padre(A,P), 
    padre(P,B). 
\end{lstlisting}

Lorem ipsum dolor sit amet, consectetur adipiscing elit, sed do eiusmod tempor incididunt ut labore et dolore magna aliqua.

   \lstset{language=java}
\begin{lstlisting}[caption= C\'odigo JAVA de una clase, label=codj]
class <Nombre>{
   public static void main(String[] args){
      instrucciones;
   }
}
\end{lstlisting}








\subsection{Entornos}
Ahora presentamos los entornos Ejemplo \ref{ejemplo1}, Solución \ref{solucion1}, Prueba \ref{prueba1}, Definición \ref{definir1}, Demostración \ref{demo1}, Observación \ref{obs1}.

\begin{ejemplo}\label{ejemplo1}
\lipsum[1] %reemplazar esta linea
\end{ejemplo}

\begin{solucion}\label{solucion1}
\lipsum[1] %reemplazar esta linea
\end{solucion}

\begin{definir}\label{definir1}
\lipsum[1] %reemplazar esta linea
\end{definir}

\begin{prueba}\label{prueba1}
\lipsum[1] %reemplazar esta linea
\end{prueba}

\begin{demo} \label{demo1}
\lipsum[1] %reemplazar esta linea
\end{demo}

\begin{obs}\label{obs1}
\lipsum[1] %reemplazar esta linea
\end{obs}

\subsection{Tablas}
Por otro lado también se muestra un ejemplo de Tabla \ref{tabla1} llenada, además de la Tabla \ref{tabla2} que tiene 3 columnas y filas con barras, la Tabla \ref{tabla3} que tiene 4 filas y 4 columnas sin barras, por último la Tabla \ref{tabla4} posee títulos que usan más de una columna y fila. 

\begin{table}[hbt]
\begin{center}
\begin{tabular}{|l|p{10cm}|}\hline
\multicolumn{2}{|c|}{\textbf{Información general}}\\
\hline
\textbf{Modelo de desarrollo}&desarrollo	Software libre y código abierto\\
\textbf{Última versión estable}&Kernel: 4.11.3 (info) 25 de mayo de 2017 (10 días)\\
\textbf{Última versión en pruebas}&	4.12.rc2 (info) 22 de mayo de 2017 (13 días)\\
\textbf{Escrito en}&	C\\
\textbf{Núcleo}&	Núcleo Linux\\
\textbf{Plataformas soportadas}	& DEC Alpha, ARM, AVR32, Blackfin, ETRAX CRIS, FR-V, H8/300, Itanium, M32R, m68k, Microblaze, MIPS, MN103, PA-RISC, PowerPC, s390, S+core, SuperH, SPARC, TILE64, Unicore32, x86, Xtensa\\
\textbf{Licencia}	&GNU General Public License y otras\\
\textbf{Estado actual}	&En desarrollo\\
\textbf{En español}	&Sí\\
\hline
\end{tabular}
\end{center}
\caption{Información General de GNU/Linux}
\label{tabla1}
\end{table}


\begin{table}[hbt]
    \centering
    \begin{tabular}{|c|c|c|}
    \hline
         &  & \\\hline
         &  & \\\hline
         &  & \\\hline
    \end{tabular}
    \caption{Ejemplo tabla con barras}
    \label{tabla2}
\end{table}


\begin{table}[htb]
    \centering
    \begin{tabular}{cccc}
        1 &2  & 3 & 4\\
        5 & 6 & 7 & 8\\
       9  & 10 &11 & 12\\
    \end{tabular}
    \caption{Ejemplo tabla con barras}
    \label{tabla3}
\end{table}

\begin{table}[htbp]
 \centering
 \begin{tabular}{|c|c|c|c|c|c|}\hline
  \textbf{S}&\textbf{SCT} &\textbf{Asignatura}&\multicolumn{2}{|c|}{\textbf{Total Horas}}&\textbf{Previatura} \\\cline{4-5}
&&&\textbf{TP}&\textbf{TA}&\\\hline
\multirow{5}{*}&a&b&c&d&r\\\hline
\end{tabular}
    \caption{Títulos de varias columnas y filas}
    \label{tabla4}
\end{table}



\subsection{Formulas Matemáticas}
A continuación se presentan entornos matemáticos con la Ecuación \ref{ecuacion1} y la Ecuación \ref{ecuacion2}.

\begin{equation}\label{ecuacion1}
C_L=\frac{(S_{22}-\delta S_{11}^*)^*}{|\varPi S_{22}|^2=-|\pi|^2}
\end{equation}
    
\begin{equation}\label{ecuacion2}
R_S=\frac{\sqrt{1-g_s}\cdot (1-|S_{11}|^2)}{1-(1-g_s)\cdot|S_{11}|^2}
\end{equation}

\subsection{Items, Descripciones y Enumeraciones}
Ejemplo de \texttt{itemize}:

  \begin{itemize}
  \item Item sin números
    \begin{itemize}
  \item nivel 2
    \begin{itemize}
    \item nivel 3
    \end{itemize}
  \end{itemize}
\end{itemize}

Ejemplo de \texttt{enumerate}:
\begin{enumerate}
  \item Item Numerado
    \begin{enumerate}
     \item Nivel 2
    \begin{enumerate}
      \item Nivel 3
    \end{enumerate}
  \end{enumerate}
\end{enumerate}

Ejemplo de \texttt{description}:
\begin{description}
  \item[Descripción] Texto descrito
  \begin{description}
  \item[Nivel 2] Texto
\end{description}
\end{description}

Ejemplo de uso intercalado

\begin{description}
  \item[Descripción] Texto descrito
  \begin{itemize}
  \item Nivel 2
  \begin{enumerate}
      \item Nivel 3
    \end{enumerate}
\end{itemize}
\end{description}



\subsection{Figuras}
En la Figura \ref{foto1} se muestra el logo de la Universidad. En cambio en la Figura \ref{figuras} se pueden apreciar 3 imágenes, la primera sería la Figura \ref{fig1}, la segunda la Figura \ref{fig2} y la tercera la Figura \ref{fig3}. 

\begin{figure}[htb]
    \centering
    \includegraphics[width=0.5\linewidth]{images/Logo-ULagos.png}
    \caption{Logo Universidad de Los Lagos}
    \label{foto1}
\end{figure}

\begin{figure}[htb]
\centering
\begin{subfigure}{0.31\textwidth}\centering
    \includegraphics[width=\textwidth]{images/Logo-ULagos.png}
    \caption{Primera figura}
    \label{fig1}
\end{subfigure}
\hfill
\begin{subfigure}{0.31\textwidth}\centering
    \includegraphics[width=\textwidth]{images/Logo-ULagos.png}
    \caption{Segunda figura}
    \label{fig2}
\end{subfigure}
\hfill
\begin{subfigure}{0.31\textwidth}\centering
    \includegraphics[width=\textwidth]{images/Logo-ULagos.png}
    \caption{Tercera figura}
    \label{fig3}
\end{subfigure}
        
\caption{Insertar subfiguras en \LaTeX.}
\label{figuras}
\end{figure}

\begin{wrapfigure}{R}{0.5\textwidth}
  \begin{center}
    \includegraphics[width=8cm]{images/Logo-ULagos.png}
  \end{center}
  \caption{Foto entre texto}
\label{entretexto}
\end{wrapfigure}
A continuación se presenta la Figura \ref{entretexto} entre texto, esta figura debe estar antes del texto y la ubicación puede ser L: izquierda; C: centrado; R: derecha. 
\lipsum[1]

